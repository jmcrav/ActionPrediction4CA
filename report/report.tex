%\documentclass[10pt,twocolumn,letterpaper]{article}
%\documentclass[review]{cvpr}
%\documentclass[rebuttal]{cvpr}
\documentclass[final]{cvpr}

\def\cvprPaperID{****} % *** Enter the CVPR/ICCV Paper ID here
\def\confName{ICCV}
\def\confYear{2021}

\usepackage{times}
\usepackage{epsfig}
\usepackage{graphicx}
\usepackage{amsmath}
\usepackage{amssymb}
\usepackage{colortbl}

% Include other packages here, before hyperref.

% If you comment hyperref and then uncomment it, you should delete
% egpaper.aux before re-running latex.  (Or just hit 'q' on the first latex
% run, let it finish, and you should be clear).
\usepackage[pagebackref=true,breaklinks=true,letterpaper=true,colorlinks,bookmarks=false]{hyperref}

% \cvprfinalcopy % *** Uncomment this line for the final submission

\def\cvprPaperID{****} % *** Enter the CVPR Paper ID here
\def\httilde{\mbox{\tt\raisebox{-.5ex}{\symbol{126}}}}

% Pages are numbered in submission mode, and unnumbered in camera-ready
%\ifcvprfinal\pagestyle{empty}\fi
\begin{document}
	
	%%%%%%%%% TITLE
	\title{Action prediction $\ldots$}
	
	\author{Giovanni Cravero\\
		Politecnico di Torino\\
		{\tt\small s193598@studenti.polito.it}
		\and
		Eugenio Dosualdo\\
		Politecnico di Torino\\
		{\tt\small s271880@studenti.polito.it}
		\and
		Marco Rondina\\
		Politecnico di Torino\\
		{\tt\small s280096@studenti.polito.it}
	}
	
	\maketitle
	%\thispagestyle{empty}
	
	%%%%%%%%% ABSTRACT
	\begin{abstract}
		Last but not least (this is the last piece of work)
	\end{abstract}
	
	%%%%%%%%% BODY TEXT
	\section{Introduction}
	Brief description of the task and the goals.
	
	%------------------------------------------------------------------------
	\section{Background}
	Introduction
	\subsection{SIMMC}
	
	\subsubsection{Data description}
	
	\subsection{BERT and Transformers}
	
	%------------------------------------------------------------------------
	\section{Model}
	What and why
	
	\subsection{Input manipulation}
	Here we can describe the tensor dataset structure and the tokenized input value, \eg

	\begin{table}[h]
		\centering

		\begin{tabular}{lcccccc}
			\rowcolor[gray]{0.9}
			& [CLS] & Q1 & L1 & & & \\
			\rowcolor[gray]{0.97}
			& [CLS] & Q1 & A1 & [SEP] & Q2 & L2 \\
			\rowcolor[gray]{0.9}
			& [CLS] & Q2 & A2 & [SEP] & Q3 & L3 \\
			\rowcolor[gray]{0.97}
			& [CLS] & Q3 & A3 & [SEP] & Q4 & L4 \\
		\end{tabular}
		
		\caption{\small{Sentences composition}}
		\label{tab:sentcomp}
	\end{table}

	The tab. \ref{tab:sentcomp} $\ldots$
		
	\subsection{Added layers - Activation functions}
	
	\subsection{Loss function}
	
	\subsubsection{Actions}
	\subsubsection{Attributes}
	
	\subsection{Tuning}
	\begin{itemize}
		\item epochs
		\item batches
		\item learning rate
		\item $\ldots$
	\end{itemize}

	%------------------------------------------------------------------------
	\section{Results}

	%------------------------------------------------------------------------
	\section{Conclusion}
	
	
	{\small
		\bibliographystyle{ieee}
		\bibliography{egbib}
	}
	
\end{document}
